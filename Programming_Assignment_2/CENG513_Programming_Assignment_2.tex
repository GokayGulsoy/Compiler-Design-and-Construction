\documentclass[16pt]{article}
\usepackage{graphicx} % Required for inserting images
\usepackage[utf8]{inputenc}
\usepackage{float}
\floatstyle{plaintop}
\usepackage{geometry}
\usepackage[skip=10pt plus1pt,indent=50pt]{parskip}
\usepackage{blindtext}
\usepackage{fancyhdr}
\usepackage[T1]{fontenc}
\usepackage{biblatex}
\usepackage{mathptmx}
\usepackage{xcolor}
\usepackage{listings}
\usepackage{cleveref}
\usepackage{lastpage} % used to display we are located in which page 
                      % out of all pages in report

\topmargin 0.1in % giving a margin from the top of page
\headsep 20pt    % giving seperation after the headers
\addbibresource{references.bib}

\renewcommand{\lstlistingname}{Code}
\renewcommand{\lstlistlistingname}{List of Code}
\lstdefinestyle{chstyle}{ % styling code listings for C language 
backgroundcolor=\color{gray!12},
basicstyle=\ttfamily\small,
commentstyle=\color{green!60!black},
keywordstyle=\color{magenta},
stringstyle=\color{blue!50!red},
showstringspaces=false,
captionpos=b,
numbers=left,
numberstyle=\footnotesize\color{gray},
numbersep=10pt,
%stepnumber=2,
tabsize=2,
frame=L,
framerule=1pt,
rulecolor=\color{red},
breaklines=true,
inputpath=code
}

\title{\bf{ Izmir Institute of Technology \\ Computer Engineering Department \\ CENG513 Programming Assignment 2}}
\author{Student Name: Gökay Gülsoy Student No: 270201072}
\date{\today}
\graphicspath{Images/}
\definecolor{myGray}{RGB}{226, 222, 222}

\begin{document}
\pagecolor{myGray}
\pagestyle{fancy}
% setting the head rule width and footer rule width
\renewcommand{\headrulewidth}{1pt}
\renewcommand{\footrulewidth}{1pt}
\renewcommand{\headruleskip}{2mm}
\renewcommand{\footruleskip}{2mm}

\fancyhead{} % clear all header fields
\fancyhead[L]{\bf CENG513 Compiler Design and Construction}
\fancyfoot{} % clear all footer fields
\fancyfoot[L]{\thepage\ of \pageref{LastPage}}
\fancyfoot[C]{\bf Extending Kaleidoscope Language}
\fancyfoot[R]{\bf Programming Assignment 2}

\maketitle
\begin{figure}[H]
    \centering
    \includegraphics[scale = 0.45]{Images/iyte_logo.png}
\end{figure}

\newpage
\lstlistoflistings
\newpage


\section{Adding New Tokens}
In order to add support for recognizing and parsing repeat-until loop construct, I have modified the lexer\cite{1.Kaleidoscope-Kaleidoscope-Introduction-and-the-Lexer-LLVM19.0.0git-documentation} and parser\cite{2.Kaleidoscope-Implementing-a-Parser-and-AST-LLVM-19.0.0git-documentation} components of Kaleidoscope language. First modification I have done is on Token enum. I have introduced to new tokens which are \textbf{repeat} and \textbf{until} which can be seen in the code listing \ref{new-tokens}.

\vspace{3.5pt}
\begin{lstlisting}[caption= Token for repeat and until added,label=new-tokens,
style=chstyle,language=C++]
//=========================================
enum Token {
  tok_eof = -1,

  // commands
  tok_def = -2,
  tok_extern = -3,

  // primary
  tok_identifier = -4,
  tok_number = -5,
  // newly added tokens for supporting
  // repeat expression until expression construct
  tok_repeat = -6,
  tok_until = -7,
};
//============================================
\end{lstlisting}


\section{Modifying Lexer}
Second modification I have done is on lexer. I have added two new equality comparisons for new tokens repeat and until as can be seen in the code listing \ref{lexer-modification}.

\vspace{2.5pt}
\begin{lstlisting}[caption= Recognizing new tokens repeat and until,label=lexer-modification,
style=chstyle,language=C++]
//=========================================
static int gettok() {
  static int LastChar = ' ';

  // Skip any whitespace.
  while (isspace(LastChar))
    LastChar = getchar();

  if (isalpha(LastChar)) { // identifier: [a-zA-Z][a-zA-Z0-9]*
    IdentifierStr = LastChar;
    while (isalnum((LastChar = getchar())))
      IdentifierStr += LastChar;

    if (IdentifierStr == "def")
      return tok_def;
    if (IdentifierStr == "extern")
      return tok_extern;

    // modifying lexer to recognize 
    // repeat and until tokens
    if (IdentifierStr == "repeat")
      return tok_repeat; 

    if (IdentifierStr == "until")
      return tok_until;

    return tok_identifier;
  }

  if (isdigit(LastChar) || LastChar == '.') { // Number: [0-9.]+
    std::string NumStr;
    do {
      NumStr += LastChar;
      LastChar = getchar();
    } while (isdigit(LastChar) || LastChar == '.');

    NumVal = strtod(NumStr.c_str(), nullptr);
    return tok_number;
  }

  if (LastChar == '#') {
    // Comment until end of line.
    do
      LastChar = getchar();
    while (LastChar != EOF && LastChar != '\n' && LastChar != '\r');

    if (LastChar != EOF)
      return gettok();
  }

  // Check for end of file.  Don't eat the EOF.
  if (LastChar == EOF)
    return tok_eof;

  // Otherwise, just return the character as its ascii value.
  int ThisChar = LastChar;
  LastChar = getchar();
  return ThisChar;
}
//============================================
\end{lstlisting}

   
\section{Adding New AST Expression for Repeat Until Loop Construct}
Third modification done was adding a new AST expression which is  \textbf{RepeatUntilExprAST} in order to represent parsed repeat-until construct. Newly added AST expression can be seen as in the code listing \ref{RepeatUntilExprAST}.

\vspace{2.5pt}
\begin{lstlisting}[caption= Adding RepeatUntilExprAST for representing parsed repeat-until loop construct,label=RepeatUntilExprAST,
style=chstyle,language=C++]
//=========================================
// RepeatUntilExprAST - This class represents a repeat-until looping 
// construct
class RepeatUntilExprAST : public ExprAST {
  std::unique_ptr<ExprAST> Body;
  std::unique_ptr<ExprAST> Condition;

  public: 
    RepeatUntilExprAST(std::unique_ptr<ExprAST> Body, std::unique_ptr<ExprAST> Condition) : 
    Body(std::move(Body)),Condition(std::move(Condition)) {}
};
//============================================
\end{lstlisting}


\section{Modifying ParsePrimary Function}
Fourth modification I have done was to extend the \textbf{ParsePrimary} function to call the \textbf{ParseRepeatExpr} function whenever \textbf{tok\_repeat} token is encountered. Modified ParsePrimary function can be seen in the code listing \ref{modified-ParsePrimary-function}. 

\vspace{2.5pt}
\begin{lstlisting}[caption=Modifying ParsePrimary function to add parsing support for repeat-until construct, label=modified-ParsePrimary-function,
style=chstyle,language=C++]
//=========================================
/// primary
///   ::= identifierexpr
///   ::= numberexpr
///   ::= parenexpr
///   ::= repeatexpr
static std::unique_ptr<ExprAST> ParsePrimary() {
  switch (CurTok) {
  case tok_identifier:
    return ParseIdentifierExpr();
  case tok_number:
    return ParseNumberExpr();
    
  // adding Parsing support for repeat-until construct
  case tok_repeat: 
    return ParseRepeatExpr();

  case '(':
    return ParseParenExpr();
  default:
    return LogError("unknown token when expecting an expression");
  }
}
//============================================
\end{lstlisting}


\section{Adding ParseRepeatExpr Function}
Fifth and final modification I have done was to implement \textbf{ParseRepeatExpr} function. This function is used to correctly parse repeat-until looping expressions. Implementation of ParseRepeatExpr function can be seen as given in the code listing \ref{ParseRepeatExpr-function}.

\vspace{2.5pt}
\begin{lstlisting}[caption=ParseRepeatExpr function parses the repeat-until loop construct, label=ParseRepeatExpr-function,
style=chstyle,language=C++]
//=========================================
/// repeatexpr ::= repeat expression until expression
static std::unique_ptr<ExprAST> ParseRepeatExpr() {
  getNextToken(); // eat the repeat
  auto Body = ParseExpression();

  if (!Body)
    return nullptr;

  if (CurTok != tok_until)
    return LogError("expected 'until' after repeat loop body");  

  getNextToken(); // eat the until
  auto Condition = ParseExpression();

  if (!Condition) 
    return nullptr;


  fprintf(stderr,"Parsed a repeat-until expr\n");
  return std::make_unique<RepeatUntilExprAST>(std::move(Body),std::move(Condition));
}
//============================================
\end{lstlisting}

\newpage
\section*{Some outputs from Execution}
Some outputs from the execution of the Kaleidoscope can be seen in the figure \ref{execution-outputs}.

\begin{figure}[!h]
    \centering
    \includegraphics[width=1.0\linewidth]{Images/Some_outputs.png}
    \caption{Some Execution Outputs}
    \label{execution-outputs}
\end{figure}

   
\clearpage
\hrule 
\printbibliography[title={References}] % Print the bibliography

\end{document}
