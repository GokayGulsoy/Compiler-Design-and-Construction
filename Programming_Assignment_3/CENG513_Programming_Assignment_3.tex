\documentclass[16pt]{article}
\usepackage{graphicx} % Required for inserting images
\usepackage[utf8]{inputenc}
\usepackage{float}
\floatstyle{plaintop}
\usepackage{geometry}
\usepackage[skip=10pt plus1pt]{parskip}
\usepackage{blindtext}
\usepackage{fancyhdr}
\usepackage[T1]{fontenc}
\usepackage
[sorting=none]
{biblatex}
\usepackage{mathptmx}
\usepackage{xcolor}
\usepackage{listings}
\usepackage{cleveref}
\usepackage{setspace}
\usepackage{lastpage} % used to display we are located in which page 
                      % out of all pages in report

\topmargin 0.1in % giving a margin from the top of page
\headsep 20pt    % giving seperation after the headers
\addbibresource{references.bib}

\renewcommand{\lstlistingname}{Code}
\renewcommand{\lstlistlistingname}{List of Code}
\lstdefinestyle{chstyle}{ % styling code listings for C language 
backgroundcolor=\color{gray!12},
basicstyle=\ttfamily\small,
commentstyle=\color{green!60!black},
keywordstyle=\color{magenta},
stringstyle=\color{blue!50!red},
showstringspaces=false,
captionpos=b,
numbers=left,
numberstyle=\footnotesize\color{gray},
numbersep=10pt,
%stepnumber=2,
tabsize=2,
frame=L,
framerule=1pt,
rulecolor=\color{red},
breaklines=true,
inputpath=code
}

\title{\bf{ Izmir Institute of Technology \\ Computer Engineering Department \\ CENG513 Programming Assignment 3}}
\author{Student Name: Gökay Gülsoy Student No: 270201072}
\date{\today}
\graphicspath{Images/}
\definecolor{myGray}{RGB}{226, 222, 222}
\onehalfspacing
\addbibresource{references.bib}

\begin{document}
\pagecolor{myGray}
\pagestyle{fancy}
% setting the head rule width and footer rule width
\renewcommand{\headrulewidth}{1pt}
\renewcommand{\footrulewidth}{1pt}
\renewcommand{\headruleskip}{2mm}
\renewcommand{\footruleskip}{2mm}

\fancyhead{} % clear all header fields
\fancyhead[L]{\bf CENG513 Compiler Design and Construction}
\fancyfoot{} % clear all footer fields
\fancyfoot[L]{\thepage\ of \pageref{LastPage}}
\fancyfoot[C]{\bf Writing an LLVM Pass}
\fancyfoot[R]{\bf Programming Assignment 3}

\maketitle
\begin{figure}[H]
    \centering
    \includegraphics[scale = 0.45]{Images/iyte_logo.png}
\end{figure}

\newpage
\lstlistoflistings
\newpage

\section{LLVM Pass Implementation}
In order to implement my own LLVM pass which replaces all integer multiplications by 2 with addition by itself, I have followed the steps in order which are indicated as given below \cite{githubBanachspacellvmtutor} :

\begin{enumerate}
    \item Adding new \textbf{MulToAdd.cpp} file under llvm-tutor/lib
    \item Updating \textbf{CMakeLists.txt} file under the llvm-tutor/lib
    \item Adding \textbf{MulToAdd.h} file under llvm-tutor/include
    \item export LLVM\_DIR=/home/gokay/llvm-project/build
    \item \$LLVM\_DIR/bin/clang -emit-llvm -S llvm-tutor/inputs/input\_for\_mul\_to\_add.c -o input\_for\_mul\_to\_add.ll
    \item \$LLVM\_DIR/bin/opt -load-pass-plugin=build/lib/libMultToAdd.so -passes="mul-to-add" -S input\_for\_mult\_to\_add.ll -o optimized\_mul\_to\_add.ll 
\end{enumerate}

\subsection{Adding MulToAdd.cpp File }
First step is to write an actual LLVM pass code which is going to traverse all instructions in basic block. I added this file under \textbf{llvm-tutor/lib} directory \cite{2019LLVMDevelopersMetting}. Implementation of MultToAdd.cpp is given as follows:

\vspace{2.5pt}
\begin{lstlisting}[caption=Implementation of MulToAdd.cpp, label=mul-to-add-implementation,
style=chstyle,language=C++]
#include "MulToAdd.h"

#include "llvm/ADT/Statistic.h"
#include "llvm/IR/IRBuilder.h"
#include "llvm/IR/Value.h"
#include "llvm/Passes/PassBuilder.h"
#include "llvm/Passes/PassPlugin.h"
#include "llvm/Transforms/Utils/BasicBlockUtils.h"

#include <random>

using namespace llvm;

#define DEBUG_TYPE "mul-to-add"
STATISTIC(SubstCount, "The # of substituted instructions");

//-----------------------------------------------------------
// MulToAdd implementation
//-----------------------------------------------------------
bool MulToAdd::runOnBasicBlock(BasicBlock &BB) {
    bool Changed = false; 

    // iterating over all instructions in basic block 
    for (auto Inst = BB.begin(), IE = BB.end(); Inst != IE; Inst++) {

        // we skip non-binary instructions 
        auto *BinOp = dyn_cast<BinaryOperator>(Inst);
        if (!BinOp) {
            continue;
        }

        // skip instructions other than mul
        unsigned OpCode = BinOp->getOpcode();
        if (OpCode != Instruction::Mul || !BinOp->getType()->isIntegerTy()) {
            
            continue;
        }

        auto *Op0 = BinOp->getOperand(0);
        auto *Op1 = BinOp->getOperand(1);

        ConstantInt *CI;
        Value *OperandToAddByItself; 
        // determining which operand is the constant integer 2
        if ((CI = dyn_cast<ConstantInt>(Op0)) && CI->equalsInt(2)) {
            OperandToAddByItself = Op1;
        }

        else if ((CI = dyn_cast<ConstantInt>(Op1)) && CI->equalsInt(2)) {
            OperandToAddByItself = Op0; 
        }

        else  {
            continue; 
        }

        // Creating new multiplication instruction 
        // and replacing it with previous instruction in basic block
        IRBuilder<> Builder(BinOp);
        Instruction *NewInst = BinaryOperator::CreateAdd(
                              OperandToAddByItself,
                              OperandToAddByItself);

        ReplaceInstWithInst(&BB,Inst,NewInst);
        Changed = true;

        // updating the statistics 
        SubstCount++;
    }

    return Changed;
}

PreservedAnalyses MulToAdd::run(llvm::Function &F,
                                llvm::FunctionAnalysisManager &) {
        
        bool Changed = false;
        
        for (auto &BB : F) {
            Changed |= runOnBasicBlock(BB);
        }

        return (Changed ? llvm::PreservedAnalyses::none() : 
                          llvm::PreservedAnalyses::all());                        
}

//-----------------------------------------------------------
// Registering our Pass
//-----------------------------------------------------------
llvm::PassPluginLibraryInfo getMulToAddPluginInfo() {
  return {LLVM_PLUGIN_API_VERSION, "mul-to-add", LLVM_VERSION_STRING,
          [](PassBuilder &PB) {
            PB.registerPipelineParsingCallback(
                [](StringRef Name, FunctionPassManager &FPM,
                   ArrayRef<PassBuilder::PipelineElement>) {
                  if (Name == "mul-to-add") {
                    FPM.addPass(MulToAdd());
                    return true;
                  }
                  return false;
                });
          }};
}

extern "C" LLVM_ATTRIBUTE_WEAK ::llvm::PassPluginLibraryInfo
llvmGetPassPluginInfo() {
  return getMulToAddPluginInfo();
}
\end{lstlisting}

Implementation for replacing integer multiplication by 2 with addition by itself mainly consists of two parts. First part is the traversal of instructions inside the basic block and replacing desired instructions with new instruction. Second part is the registration of the pass. I give the name "mul-to-add" for my pass, it should be used with -passes option whenever pass is to applied over any .ll file 

\subsection{Updating CMakeLists.txt Under llvm-tutor/lib Directory}
I had to update the CMakeLists.txt file located under the llvm-tutor/lib directory so that my new pass is build as a plugin when I make the project. I have updated two parts in the CMakeLists.txt first is adding MulToAdd under LLVM\_TUTOR\_PLUGINS and adding \textbf{set(MulToAdd\_SOURCES MulToAdd.cpp)} line below other set commands. Updated CMakeLists.txt file is as follows: 

\vspace{2.5pt}
\begin{lstlisting}[caption=Updated CMakeLists.txt file located under llvm-tutor/lib, label=update-CMakeLists.txt-under-llvm-tutor/lib,
style=chstyle,language=C++]
# THE LIST OF PLUGINS AND THE CORRESPONDING SOURCE FILES
# ======================================================
set(LLVM_TUTOR_PLUGINS
    StaticCallCounter
    DynamicCallCounter
    FindFCmpEq
    ConvertFCmpEq
    InjectFuncCall
    MBAAdd
    MBASub
    RIV
    DuplicateBB
    OpcodeCounter
    MergeBB
    MulToAdd
    )

set(StaticCallCounter_SOURCES
  StaticCallCounter.cpp)
set(DynamicCallCounter_SOURCES
  DynamicCallCounter.cpp)
set(FindFCmpEq_SOURCES
  FindFCmpEq.cpp)
set(ConvertFCmpEq_SOURCES
  ConvertFCmpEq.cpp)
set(InjectFuncCall_SOURCES
  InjectFuncCall.cpp)
set(MBAAdd_SOURCES
  MBAAdd.cpp)
set(MBASub_SOURCES
  MBASub.cpp)
set(RIV_SOURCES
  RIV.cpp)
set(DuplicateBB_SOURCES
  DuplicateBB.cpp)
set(OpcodeCounter_SOURCES
  OpcodeCounter.cpp)
set(MergeBB_SOURCES
  MergeBB.cpp)
set(MulToAdd_SOURCES  
   MulToAdd.cpp)

# CONFIGURE THE PLUGIN LIBRARIES
# ==============================
foreach( plugin ${LLVM_TUTOR_PLUGINS} )
    # Create a library corresponding to 'plugin'
    add_library(
      ${plugin}
      SHARED
      ${${plugin}_SOURCES}
      )

    # Configure include directories for 'plugin'
    target_include_directories(
      ${plugin}
      PRIVATE
      "${CMAKE_CURRENT_SOURCE_DIR}/../include"
    )

    # On Darwin (unlike on Linux), undefined symbols in shared objects are not
    # allowed at the end of the link-edit. The plugins defined here:
    #  - _are_ shared objects
    #  - reference symbols from LLVM shared libraries, i.e. symbols which are
    #    undefined until those shared objects are loaded in memory (and hence
    #    _undefined_ during static linking)
    # The build will fail with errors like this:
    #    "Undefined symbols for architecture x86_64"
    # with various LLVM symbols being undefined. Since those symbols are later
    # loaded and resolved at runtime, these errors are false positives.
    # This behaviour can be modified via the '-undefined' OS X linker flag as
    # follows.
    target_link_libraries(
      ${plugin}
      "$<$<PLATFORM_ID:Darwin>:-undefined dynamic_lookup>"
      )
endforeach()
\end{lstlisting}

\subsection{Adding MulToAdd.h file Under llvm-tutor/include Directory}
I have added \textbf{MulToAdd.h} file under llvm-tutor/include directory. MulToAdd.h file contains MulToAdd struct along with required methods to run pass over the basic block. MulToAdd.h file is as follows: 

\vspace{2.5pt}
\begin{lstlisting}[caption=MulToAdd.h Header File, label=mul-to-add-header-file,
style=chstyle,language=C++]
//========================================================================
// FILE:
//    MulToAdd.h
//
// DESCRIPTION:
//    Declares the MulToAdd pass for pass manager.
//========================================================================
#ifndef LLVM_TUTOR_MUL_TO_ADD_H
#define LLVM_TUTOR_MUL_TO_ADD_H

#include "llvm/IR/PassManager.h"
#include "llvm/Pass.h"

struct MulToAdd : public llvm::PassInfoMixin<MulToAdd> {
  llvm::PreservedAnalyses run(llvm::Function &F,
                              llvm::FunctionAnalysisManager &);
  bool runOnBasicBlock(llvm::BasicBlock &B);

  // Without isRequired returning true, this pass will be skipped for functions
  // decorated with the optnone LLVM attribute. Note that clang -O0 decorates
  // all functions with optnone.
  static bool isRequired() { return true; }
};
#endif
\end{lstlisting}

run function takes reference to Function type and reference to FunctionAnalysisManager as it's parameters. runOnBasicBlock function takes a reference to BasicBlock over which pass will be applied. isRequired method should return true, otherwise pass will be skipped for functions 
decorated with optnone LLVM attribute. By default when source code is compiled with clang -O0, all functions are decorated with optnone \cite{writingLLVMPass}.

\subsection{Generating IR file and Running Pass over the Input File}
As llvm-tutor is an out-of-tree project, I had to specify path for my LLVM installation directory with the command \textbf{export LLVM\_DIR=/home/gokay/llvm-project/build} so that LLVM\_DIR environment variable holds the LLVM installation directory for the current shell session. Then I have generated IR for the input file with the command \textbf{\$LLVM\_DIR/bin/clang -emit-llvm -S llvm-tutor/inputs/input\_for\_mul\_to\_add.c -o input\_for\_mul\_to\_add.ll}. Finally, I run the pass over the IR with the command \textbf{\$LLVM\_DIR/bin/opt -load-pass-plugin=build/lib/libMulToAdd.so -passes="mul-to-add" -S input\_for\_mult\_to\_add.ll -o optimized\_mul\_to\_add.ll} to generate the optimized IR file. As a result of running the pass, I got the optimized version of the IR file which is named as \textbf{optimized\_mul\_to\_add.ll}. initial and optimized IR files are as follows: 

\vspace{2.5pt}
\begin{lstlisting}[caption=Initial IR File, label=input_for_mul_to_add.ll,
style=chstyle,language=C++]
; ModuleID = '../inputs/input_for_mul_to_add.c'
source_filename = "../inputs/input_for_mul_to_add.c"
target datalayout = "e-m:e-p270:32:32-p271:32:32-p272:64:64-i64:64-i128:128-f80:128-n8:16:32:64-S128"
target triple = "x86_64-unknown-linux-gnu"

; Function Attrs: noinline nounwind optnone uwtable
define dso_local i32 @main(i32 noundef %0, ptr noundef %1) #0 {
  %3 = alloca i32, align 4
  %4 = alloca i32, align 4
  %5 = alloca ptr, align 8
  %6 = alloca i32, align 4
  %7 = alloca i32, align 4
  %8 = alloca i32, align 4
  %9 = alloca i32, align 4
  %10 = alloca i32, align 4
  %11 = alloca i32, align 4
  %12 = alloca i32, align 4
  %13 = alloca i32, align 4
  store i32 0, ptr %3, align 4
  store i32 %0, ptr %4, align 4
  store ptr %1, ptr %5, align 8
  %14 = load ptr, ptr %5, align 8
  %15 = getelementptr inbounds ptr, ptr %14, i64 1
  %16 = load ptr, ptr %15, align 8
  %17 = call i32 @atoi(ptr noundef %16) #2
  store i32 %17, ptr %6, align 4
  %18 = load ptr, ptr %5, align 8
  %19 = getelementptr inbounds ptr, ptr %18, i64 2
  %20 = load ptr, ptr %19, align 8
  %21 = call i32 @atoi(ptr noundef %20) #2
  store i32 %21, ptr %7, align 4
  %22 = load ptr, ptr %5, align 8
  %23 = getelementptr inbounds ptr, ptr %22, i64 3
  %24 = load ptr, ptr %23, align 8
  %25 = call i32 @atoi(ptr noundef %24) #2
  store i32 %25, ptr %8, align 4
  %26 = load ptr, ptr %5, align 8
  %27 = getelementptr inbounds ptr, ptr %26, i64 4
  %28 = load ptr, ptr %27, align 8
  %29 = call i32 @atoi(ptr noundef %28) #2
  store i32 %29, ptr %9, align 4
  %30 = load i32, ptr %6, align 4
  %31 = mul nsw i32 2, %30
  store i32 %31, ptr %10, align 4
  %32 = load i32, ptr %7, align 4
  %33 = mul nsw i32 2, %32
  store i32 %33, ptr %11, align 4
  %34 = load i32, ptr %8, align 4
  %35 = mul nsw i32 2, %34
  store i32 %35, ptr %12, align 4
  %36 = load i32, ptr %9, align 4
  %37 = mul nsw i32 2, %36
  store i32 %37, ptr %13, align 4
  %38 = load i32, ptr %10, align 4
  %39 = load i32, ptr %11, align 4
  %40 = add nsw i32 %38, %39
  %41 = load i32, ptr %12, align 4
  %42 = add nsw i32 %40, %41
  %43 = load i32, ptr %13, align 4
  %44 = add nsw i32 %42, %43
  ret i32 %44
}

; Function Attrs: nounwind willreturn memory(read)
declare i32 @atoi(ptr noundef) #1

attributes #0 = { noinline nounwind optnone uwtable "frame-pointer"="all" "min-legal-vector-width"="0" "no-trapping-math"="true" "stack-protector-buffer-size"="8" "target-cpu"="x86-64" "target-features"="+cmov,+cx8,+fxsr,+mmx,+sse,+sse2,+x87" "tune-cpu"="generic" }
attributes #1 = { nounwind willreturn memory(read) "frame-pointer"="all" "no-trapping-math"="true" "stack-protector-buffer-size"="8" "target-cpu"="x86-64" "target-features"="+cmov,+cx8,+fxsr,+mmx,+sse,+sse2,+x87" "tune-cpu"="generic" }
attributes #2 = { nounwind willreturn memory(read) }

!llvm.module.flags = !{!0, !1, !2, !3, !4}
!llvm.ident = !{!5}

!0 = !{i32 1, !"wchar_size", i32 4}
!1 = !{i32 8, !"PIC Level", i32 2}
!2 = !{i32 7, !"PIE Level", i32 2}
!3 = !{i32 7, !"uwtable", i32 2}
!4 = !{i32 7, !"frame-pointer", i32 2}
!5 = !{!"clang version 19.0.0git (https://github.com/llvm/llvm-project.git 790bcecce6c135476d2551805c09ed670b9f8418)"}
\end{lstlisting}

\vspace{2.5pt}
\begin{lstlisting}[caption=Optimized IR File, label=optimized_mul_to_add.ll,
style=chstyle,language=C++]
; ModuleID = 'input_for_mul_to_add.ll'
source_filename = "../inputs/input_for_mul_to_add.c"
target datalayout = "e-m:e-p270:32:32-p271:32:32-p272:64:64-i64:64-i128:128-f80:128-n8:16:32:64-S128"
target triple = "x86_64-unknown-linux-gnu"

; Function Attrs: noinline nounwind optnone uwtable
define dso_local i32 @main(i32 noundef %0, ptr noundef %1) #0 {
  %3 = alloca i32, align 4
  %4 = alloca i32, align 4
  %5 = alloca ptr, align 8
  %6 = alloca i32, align 4
  %7 = alloca i32, align 4
  %8 = alloca i32, align 4
  %9 = alloca i32, align 4
  %10 = alloca i32, align 4
  %11 = alloca i32, align 4
  %12 = alloca i32, align 4
  %13 = alloca i32, align 4
  store i32 0, ptr %3, align 4
  store i32 %0, ptr %4, align 4
  store ptr %1, ptr %5, align 8
  %14 = load ptr, ptr %5, align 8
  %15 = getelementptr inbounds ptr, ptr %14, i64 1
  %16 = load ptr, ptr %15, align 8
  %17 = call i32 @atoi(ptr noundef %16) #2
  store i32 %17, ptr %6, align 4
  %18 = load ptr, ptr %5, align 8
  %19 = getelementptr inbounds ptr, ptr %18, i64 2
  %20 = load ptr, ptr %19, align 8
  %21 = call i32 @atoi(ptr noundef %20) #2
  store i32 %21, ptr %7, align 4
  %22 = load ptr, ptr %5, align 8
  %23 = getelementptr inbounds ptr, ptr %22, i64 3
  %24 = load ptr, ptr %23, align 8
  %25 = call i32 @atoi(ptr noundef %24) #2
  store i32 %25, ptr %8, align 4
  %26 = load ptr, ptr %5, align 8
  %27 = getelementptr inbounds ptr, ptr %26, i64 4
  %28 = load ptr, ptr %27, align 8
  %29 = call i32 @atoi(ptr noundef %28) #2
  store i32 %29, ptr %9, align 4
  %30 = load i32, ptr %6, align 4
  %31 = add i32 %30, %30
  store i32 %31, ptr %10, align 4
  %32 = load i32, ptr %7, align 4
  %33 = add i32 %32, %32
  store i32 %33, ptr %11, align 4
  %34 = load i32, ptr %8, align 4
  %35 = add i32 %34, %34
  store i32 %35, ptr %12, align 4
  %36 = load i32, ptr %9, align 4
  %37 = add i32 %36, %36
  store i32 %37, ptr %13, align 4
  %38 = load i32, ptr %10, align 4
  %39 = load i32, ptr %11, align 4
  %40 = add nsw i32 %38, %39
  %41 = load i32, ptr %12, align 4
  %42 = add nsw i32 %40, %41
  %43 = load i32, ptr %13, align 4
  %44 = add nsw i32 %42, %43
  ret i32 %44
}

; Function Attrs: nounwind willreturn memory(read)
declare i32 @atoi(ptr noundef) #1

attributes #0 = { noinline nounwind optnone uwtable "frame-pointer"="all" "min-legal-vector-width"="0" "no-trapping-math"="true" "stack-protector-buffer-size"="8" "target-cpu"="x86-64" "target-features"="+cmov,+cx8,+fxsr,+mmx,+sse,+sse2,+x87" "tune-cpu"="generic" }
attributes #1 = { nounwind willreturn memory(read) "frame-pointer"="all" "no-trapping-math"="true" "stack-protector-buffer-size"="8" "target-cpu"="x86-64" "target-features"="+cmov,+cx8,+fxsr,+mmx,+sse,+sse2,+x87" "tune-cpu"="generic" }
attributes #2 = { nounwind willreturn memory(read) }

!llvm.module.flags = !{!0, !1, !2, !3, !4}
!llvm.ident = !{!5}

!0 = !{i32 1, !"wchar_size", i32 4}
!1 = !{i32 8, !"PIC Level", i32 2}
!2 = !{i32 7, !"PIE Level", i32 2}
!3 = !{i32 7, !"uwtable", i32 2}
!4 = !{i32 7, !"frame-pointer", i32 2}
!5 = !{!"clang version 19.0.0git (https://github.com/llvm/llvm-project.git 790bcecce6c135476d2551805c09ed670b9f8418)"}
\end{lstlisting}

If we look at the initial and optimized IR files, specifically at lines \textbf{42-52} we can see that in initial IR file, mul instructions are used to multiply by two, whereas in optimized IR file those mul instructions are replaced by add instructions to perform addition of integer by itself.

\subsection{Effectiveness of Pass, Potential Limitations and Improvements}
In the current version my LLVM pass it can find all integer multiplications inside functions, and replace them with addition by itself.
However it is restricted to functions, meaning that it can only work with integer variables inside functions not with variables defined in global scope, also it is restricted to integer type it can not perform replacement for other numeric data types (float, double, etc.) for now. Improvements that can be done to extend the functionality of the pass is to adding support for other numeric data types, and performing replacement even when variables are not defined inside functions (in global scope). 

\newpage
\hrule
\printbibliography[title={References}] % Print the bibliography


\end{document}
